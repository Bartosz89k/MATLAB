\subsection{Example Problems}

\begin{enumerate}

\item Solve the ladder problem using Grid Search, Newton-Raphson and
  Gradient search. Plot the length of the ladder as a function of the
  slope. 

\item In MATLAB type {\it help fminbnd} and {\it help fminsearch}. Use
  the help function to learn how to use these functions and use them
  to solve the ladder problem. Compare your results to your own
  numerical method. 

\item Solve the Density and Temperature problem using Grid Search and
  Gradient Search.

\item You are asked to design a covered conical pit to store 50 $m^3$
  of waste liquid. Assume excavation costs at $\$100/m^3$, side lining
  costs at $\$50/m^2$, and cover costs at $25/m^2$. Determine the
  dimensions of the pit that minimizes cost. If the slope is
  unconstrained and if the side slope must be less than 45$^o$. Here
  are some equations that might help you

  \beq
  \begin{matrix}
  Excavation~Volume = \pi r^2 h/3\\
  Side~Lining~Area = \pi r\sqrt{h^2+r^2}\\
  Cover~Area = \pi r^2\\
  \end{matrix}
  \eeq

  Create plots of Cost vs. Radius as well as cone angle
  vs. radius. Solve the minimization problem using the built in MATLAB 
  function fminbnd. Is the constraint a limiting factor? If so, what
  is the radius with the constraint? 

\item A company makes two types of products, A and B. These products
  are produced during a 40-hr work week and then shipped out at the
  end of the week. They require 20 and 5 kg of raw material per kg of
  product, respectively, and the company has access to 9500 kg of raw
  material per week. Only one product can be created at a time with
  production times for each of 0.04 and 0.12 hr, respectively. The
  plant can only store 550 kg of total product per week. Finally, the
  company makes profits of \$45 and \$20 on each unit of A and B,
  respectively. Each unit of product is equivalent to a kg.\\
  \ \\
  Create a mesh plot of the objective function and plot all constraint
  equations on it. Evaluate all 
  limiting cases by taking a look at your Pareto Frontier. Once you
  have evaluated which limiting conditions is the minimum use the
  fmincon function like we did in class to solve for the
  solution. Note that fmincon does not exist in Octave so you will
  have to team up with someone who owns MATLAB and solve the problem
  that way. Plot a blue square on the solution of this problem just as
  we did in class. 

\item You are an engineer planning to airlift $M_t = 2000 kg$ worth of supplies
  to people in need. The supplies are being dropped at an altitude of
  $500 m$. The price of a parachute is given using the equation below

  \beq
  c_{chute} = 200 + 56l + 0.1A^2
  \eeq
  
  where $l=\sqrt{r}$,$A=2\pi r^2$ and $r$ is the radius of the
  parachute. It is possible to put all 2000 $kg$ worth of material
  into 1 parachute however the parachute must be really big because
  the velocity that the product strikes the ground must be less than
  $20 m/s$ to ensure no one is injured and the product is not
  damaged. We know from the previous section on Numerical Integration
  that a parachute in free fall is governed by the equation below

  \beq
  \ddot{z} + (c/m)\dot{z} = g
  \eeq

  where $z$ is the altitude, $c=3$ is the drag coefficient and
  $g=9.81$ is the gravitational constant on Earth. This equation can
  be solved analytically to obtain $z(t)$ to compute when the
  parachute collides with the ground. The time can then be substituted
  into the velocity equation to ensure that the velocity is under $20
  m/s$. However, it is also possible to break up the single parachute
  into multiple chutes thereby decreasing the radius of each
  parachute. The mass of each parachute than becomes $m=M_t/N$ and the
  total cost of the mission is $Cost=Nc_{chute}$. Compute the number
  of parachutes and radius of each parachute that minimizes the cost
  of the entire mission. Create plots to support your answer.

  {\bf Hint:} If you solve for the velocity analytically you should
  obtain this equation below

  \beq
  v(t) = \frac{-gm}{c}(1-e^{-ct/m})
  \eeq
  
  integrating one more time yields the position 

  \beq
  z(t) = z_0 - \frac{gm}{c}t + \frac{gm^2}{c^2}e^{-ct/m}
  \eeq

  Thus, rather than integrating the equations numerically until the
  parachute hits the ground you can turn this problem into a root
  finding problem to determine when the parachute collides with the
  ground. 

\end{enumerate}

\subsection{Project}

As you know every engineering problem can be cast into the form  \\
\ \\
dependent variable = function( independent variable,parameters)\\
\ \\
The problem is that alot of times the independent variable can
change. So the question is what is the best solution? What is the
criteria for optimal solutions? Cost? Weight? Size? It's up to you,
you are the engineer. Your task is to find an optimization problem out
in the world and figure out what the best solution is. Restrict
yourself to a 2-D problem. 1D Problems will not be accepted. I
encourage you to use fmincon in MATLAB if not you can try using a grid
search and/or try N-R however N-R in 2D is pretty complex. \\
\ \\
Your deliverable for this assignment will be to write a report
detailing your optimization problem. The sections included in your
report will be the following:
\begin{enumerate}
\item{{\bf Introduction}} 
Explain what the problems are. Why do we care? Why
is this important? Give some background on this type of problem. 
\item{{\bf Mathematical Model}}
Explain the theory on how these problems are
solved. Include equations in your report. Do not screenshot equations
or just type them in. You are engineers. It's time to learn how to use
Equation Editor. Finally, include all pertinent data required to run
your code. Are there fixed parameters that do not vary? Include them
in this section. 
\item{{\bf Results}}
Explain your inputs to your code and your outputs. Do
not copy and paste MATLAB output. Write your results in normal
english. For example, "When the weight of the cat is 5 lbs the
terminal velocity is 50 ft/s. If the weight of the cat is increased to
10 lbs the terminal velocity of the cat is 80 ft/s".  
\item{{\bf Appendix MATLAB Code}}
Copy and paste your MATLAB code. This is
the only place the word MATLAB should be. No supporting text required,
simply copy and paste your code into this section. 
\end{enumerate}
