\subsection{Arrays - Palm - Chapter 2}

\begin{enumerate}

\item \textbf{Inputting Arrays}

Whenever you type a variable into MATLAB 

$>>$ x = 1

MATLAB will create an array. The variable x currently is a 1x1
array. That is it has 1 row and 1 column. It is possible however to
create a vector of numbers.

$>>$ x = [1 2 3 4]

This will create a vector with 1 row and 4 columns. It is possible to
make the same vector with only 1 column by using a semicolon.

$>>$ x = [1 ; 2 ; 3 ; 4]

This will create a variable with 4 rows and 1 column. It is also
possible to create a column vector by using the transpose value.

$>>$x = [1 2 3 4]'

The ' will transpose the vector. In order to create a matrix you
simply combine the semicolons and spaces.

$>>$ A = [1 2 ; 3 4]

This will create a 2x2 matrix. It is also possible to create rules for
vectors. For example, say you want all even number from 2 to 100. This
can be accomplished by using the form

\item \textbf{Linspace vs. Increment}

{\it start:increment:end}

$>>$ x = 2:2:100

The line above will start at 2, increment by 2 and stop when it hits
100. You can also use the linspace command which uses the form

{\it linspace(start,end,number of elements)}

$>>$ x = linspace(1,10,5)

The line above will create a vector that starts at 1 and end at 10 but
contains 5 elements.

\item \textbf{Length Command}

$>>$ L = length(x)

Length will compute the number of elements in the vector. In this case
L will be equal to 5. If you type whos in MATLAB you will see that x
is a vector 1x5 and L is a 1x1 but its value is 5.

\item \textbf{ Creating Empty Matrices}

To create empty matrices you may use the zeros() and ones() command.

$>>$ A = zeros(5,3)

The line above will create a matrix of zeros with 5 rows and 3
columns. 

Note that adding matrices is very simple. 

$>>$ B = zeros(5,3) + 5

will create a matrix of zeros with 5 rows and 3 columns and then add
the number 5 to every element in the matrix.

\item \textbf{Add/Substract and Multiply/Divide}

$>>$ C = A + B

this will then add the contents of A to B. Similar results are seen
for substraction. Multiplication is tricky. The following line will
throw an error.

$>>$ C = A*B $\leftarrow$ ERROR

The reason is that A is a 5x3 and B is a 5x3. As such the inner matrix
dimensions do not match.

$>>$ x = [1 ; 3]

$>>$ A = [1 2 ; 4 5]

$>>$ b = A*x

the lines above are totally valid because x is a 2x1 and A is a 2x2
thus b is a 2x1. Type in whos to verify. 

\item \textbf{Dot Operator}

It may be beneficial to compute the square of the numbers from say 1
to 10.

$>>$ x = 1:1:10

will create the numbers from 1 to 10 incrementing by 1.

$>>$ x = x\textrm{\^}2 $\leftarrow$ ERROR

This again will throw an error because x is a 1x10. A 1x10 times a
1x10 does not have the inner matrix dimensions that match. In order to
multiply these matrices you must use the dot operator.

$>>$ x = x.\textrm{\^}2

You can also use the dot operator for muliplication and division.

$>>$ y = 2*ones(1,10)

$>>$ z = x./y

This will create a vector of ones (1x10) and then multiply each
element by 2. Thus y will be a vector of 2's. z will then be every
element in x divided by y. Try this out to see if it works.

\item \textbf{Reference Elements of an Array}

Finally you may find yourself trying to use the second row and third
column of a matrix. 

$>>$ A = magic(5)

will create a 5x5 matrix using the magic algorithm. If you then wish
to select the second element and third column you simply type

$>>$ a32 = A(3,2)

You can also grab the entire column by typing

$>>$ a2 = A(:,2)

which will grab the entire second column.

\item {\bf Semicolons!}

Notice that when you type

$>>$ x = 0:0.01:100

MATLAB will spit out a lot of numbers. When creating a script it is
common practice to stop the output of computation by putting a plug on
the computation. This is accomplished by using a semi-colon.x

$>>$ x = 0:0.01:100;

will not output anything to the Command Window.

\item {\bf Solving Multiple Algebraic Equations}

The task of solving the equation below is trivial to do by hand
however it is easy to make simple mistakes.

\begin{equation} \nonumber
\begin{matrix}
5x-3y+4z = 41 \\
12x + 6y -7z = -26 \\
-4x + 2y + 6z = 14 
\end{matrix}
\end{equation}

In order to do this by hand we simply type 

$>>$ A = [5 -3 4 ; 12 6 -7 ; -4 2 6]

which will create an A matrix of the coefficients of our system. Then

$>>$ b = [41;-26;14]

which will create a vector of the solutions. Finally assuming the form
Ax=b we can solve this by multiplying the left hand side by the
inverse of A.

$>>$ x = inv(A)*b

\item {\bf Evaluating Functions}

Assume you have the function $y = x^2cos(x/2)$ and you wish to
evaluate the function at x = $\pi/4$. That would simply be

$>>$ x = pi/4

$>>$ y = x\textrm{\^}2*cos(x/2)

However if you wish to evaluate the function over an interval say -pi
to pi then you need to use vector math.

$>>$ x = linspace(-pi,pi,100)

which creates a vector of 100 elements from -$\pi$ to $\pi$. Then you
can evaluate y using the dot operator.

$>>$ y = x.\textrm{\^}2.*cos(x/2)

Note that you do not need a dot operator when dividing by 2 because 2
is a scalar.

\item {\bf Other Arrays}

There are two other types of arrays. One is a cell array and the other
is a structure. I will go over them briefly here. 

A structure is a bucket that holds multiple attributes of
information. To make a structure you simply type in

$>>$ computers.name = 'Toshiba'

$>>$ computers.CPU = 3.9

This will create a structure called computers. The structure has two
attributes, name and CPU. If you would like to add another element to
the structure simply type 

$>>$ computers(2).name = 'Lenovo'

$>>$ computers(2).CPU = 2.0

You can then call the structure in two ways. You can list all
attributes of an element by typing

$>>$ computers(2)

or you can list all the names of the computers by typing

$>>$ computers.name

Finally you can access the 1 computers CPU by typing in 

$>>$ computers(1).CPU

A cell array can be created which contains any all variable type
embedded. For example

$>>$ x = [43 56 77]

$>>$ s = 'Hello'

$>>$ u = 0

$>>$ computers.name = 'Toshiba'

$>>$ computers(2).name = 'Lenovo'

creates 4 variables. The first is a double with 3 elements. The second
is a string with 5 elements. The third is a double with 1 element and
the last is a structure with 2 elements. These can be combined into a
cell array by using curly braces \{\}.

$>>$ c = \{x,s,u,computers\}

Note however that the length of c is 4. Furthermore if you wish to
access part of the cell array you need to use curly braces.

$>>$ c\{1\}

will output the contents of the first cell which in this case is x.
\ \\

{\bf Functions Learned}

magic, ones, zeros, linspace, length, inv

\end{enumerate}
