\subsection{Example Problems}

\begin{enumerate}

\item Euler's method has a unique property in that it convertes a
  continuous differential equation such as the one below 

\begin{equation}
\ddot{y} + 2\dot{y} + 4y = 0
\end{equation}

into a discrete differential equation like the form below.

\begin{equation}
\begin{matrix}
{y}_{n+1} = {y}_n + \dot{y}_n\Delta t \\
\dot{y}_{n+1} = \dot{y}_n + (-2\dot{y}_n - 4y)\Delta t
\end{matrix}
\end{equation}

We've done this problem a
million times but what we haven't done is placed Euler's method into
the following form. 

\begin{equation}
\begin{Bmatrix}
y_{n+1} \\ \dot{y}_{n+1} \end{Bmatrix} = \begin{bmatrix} 1 & \Delta t
  \\ -4\Delta t & (1-2\Delta t) \end{bmatrix} \begin{Bmatrix} y_{n}
  \\ \dot{y}_n \end{Bmatrix}
\end{equation}

In this form it is possible to use vector algebra to compute the
solution to the differential equation. 

\begin{equation}
\vec{y}_{n+1} = A\vec{y}_n
\end{equation}

First, take your function from before spring break

\textcolor{blue}{function} myEuler(deltat)

and edit it to use the matrix form of Euler's method. Do the same
thing you did last time where you vary the timestep until the graph
does not change. However, this time compute the eigenvalues of the
matrix A. What do you notice about eigenvalues of the matrix as the
timestep gets smaller? What happens to the eigenvalues when the
timestep is too big and the graph goes unstable?

\item Simulate the multi body system derived in this chapter. Use the
  following data below.

  m1 = 2;\\
  m2 = 3;\\
  k1 = 50;\\
  k2 = 100;\\
  x1(t=0) = 5;\\
  xdot1(t=0) = 0;\\
  x2(t=0) = 10;\\
  xdot2(t=0) = 0;\\
  
  You will create the solution to the differential equation using three
  different methods. 

  \begin{enumerate}

    \item RK4

    \item z(t) = expm(A*t)*z0 - The analytical solution

    \item z(t) = V*expm(L*t)*inv(V)*z0 - Eigenvalue solution

  \end{enumerate}

  where V is the eigenvalues and L is the eigenvectors. You can use
  the eig function \\
  \ \\
  For plotting, plot all velocities (xdot1 and xdot2) for all three
  solutions on the same graph. All lines should be on top of each
  other.  \\
  \ \\
  In addition plot all positions (x1 and x2) for all three solutions
  on the same graph. Again, all lines should match. \\
  \ \\
  LIST THE EIGENVALUES OF THE A MATRIX IN YOUR REPORT. Explain what
  the eigenvalues mean in your own words. How many eigenvalues are
  there? Why are there so many? How many degrees of freedom does your
  system have? 

\end{enumerate}
